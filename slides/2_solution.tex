% % % %
% % % % Solution
% % % %
\section{Solution}

\begin{frame}\frametitle{Domain knowledge: diabetic retinopathy symptoms}

\begin{itemize}
\item {\hyperlink{https://goo.gl/s5lMt8}{DR symptoms cheatsheet: https://goo.gl/s5lMt8}
\begin{center}
\par \includegraphics[interpolate=true,valign=c,width=0.4\textwidth]{pics/symptoms_pic.png}
\end{center}
Was prepared with assistance of Vera Shevchenko
}
\item \href{http://www.icoph.org/downloads/Diabetic-Retinopathy-Detail.pdf}{International Clinical Diabetic Retinopathy Disease Severity Scale, Detailed Table: \small http://www.icoph.org/downloads/Diabetic-Retinopathy-Detail.pdf}
\end{itemize}

\end{frame}

\subsection{Data preparation}


\begin{frame}\frametitle{Preprocessing}
\begin{columns}

\begin{column}{9cm}
\begin{itemize}
\item Crop black borders
\vspace{20pt}

\item Extent to square + resize
\vspace{20pt}

\item Optionally: crop internal square
\vspace{20pt}

\item Optionally: Local Contrast Normalization (LCN) \\ 
      $ \hat{I}_{x,y} = \frac{I_{x,y} - \mu_{x,y}}{\sigma_{x,y}} $
\end{itemize}
\end{column}

\begin{column}{6cm}
\begin{tabular}{ @{}c m{0.25cm} c }

	\includegraphics[valign=c,height=0.5cm]{pics/10_left.jpeg} & $\rightarrow$ &
    \includegraphics[valign=c,height=0.5cm]{pics/10_left.png}
    \vspace{10pt} \\ 
    
	\includegraphics[valign=c,height=0.5cm]{pics/10_left.png} & $\rightarrow$ &
    \includegraphics[valign=c,height=0.25cm]{pics/10_left.png}
	\vspace{10pt} \\ 	

	\adjustbox{valign=c}{\begin{tikzpicture}
	    \node[anchor=south west,inner sep=0] (image) at (0,0) {\includegraphics[valign=c,height=0.5cm]{pics/10_left.png}};
	        \begin{scope}[x={(image.south east)},y={(image.north west)}]
	            \draw[red] (0,0.5) -- (0.5,1) -- (1,0.5) -- (0.5,0) -- (0,0.5);
	        \end{scope}
	\end{tikzpicture}} & $\rightarrow$ &
	    \includegraphics[valign=c,height=0.5cm]{pics/10_left_inner.png}
    \vspace{10pt} \\

%    \includegraphics[valign=c,height=0.5cm]{pics/10_left.png} & $\rightarrow$ &
%    \includegraphics[valign=c,height=0.5cm]{pics/10_left_inner.png}
%	\vspace{10pt} \\

	\includegraphics[valign=c,height=0.5cm]{pics/10_left.png} & $\rightarrow$ &
	\includegraphics[valign=c,height=0.5cm]{pics/10_left_lcn.png}
	\\


\end{tabular}
\end{column}

\end{columns}
\end{frame}

\begin{frame}\frametitle{Augmentation}
\begin{columns}[T]
\begin{column}{5cm}
\begin{itemize}
\item Stuff that worked
\begin{itemize}
\item Vertical/horizontal Mirror 
\item Random shifts
\item Random color noise
\end{itemize}
\end{itemize}

\end{column}
\begin{column}{5cm}
\begin{itemize}
\item Stuff that not quite worked
\begin{itemize}
\item Rotations
\item Krizhevsky-style\footnotemark color augmentation
\item Scaling
\item Sheering
\item Many more...
\end{itemize}
\end{itemize}
\end{column}
\end{columns}
\vspace{0.25cm}
\par We suppose that few augmentations worked because of insufficient depth of our network, but experiments with deeper nets led to more overfitting.
\footcitetext{KRIZ}
\end{frame}

\subsection{Network configuration}

\begin{frame}\frametitle{Network configuration}
\centering
\begin{table}[]
\tiny
\centering
\begin{tabular}{@{}clcllr@{}}
\toprule
   & Layer type                & Size &             & Output Shape        & Outputs \\ \midrule
1  & InputLayer                &      &             & (64, 3, 256, 256)   & $196\,608$  \\
2  & \textbf{SliceRotateLayer} &      &             & (256, 3, 128, 128)  & $49\,152$   \\
3  & Conv2DDNNLayer            & 3x3  & LReLU       & (256, 64, 126, 126) & $1\,016\,064$ \\
4  & MaxPool2DDNNLayer         & 3x3  & stride 2x2  & (256, 64, 62, 62)   & $246\,016$  \\
5  & DropoutLayer              &      & P=0.1       & (256, 64, 62, 62)   & $246\,016$  \\
6  & Conv2DDNNLayer            & 3x3  &             & (256, 96, 60, 60)   & $345\,600$  \\
7  & MaxPool2DDNNLayer         & 3x3  & stride 2x2  & (256, 96, 29, 29)   & $80\,736$   \\
8  & DropoutLayer              &      & P=0.2       & (256, 96, 29, 29)   & $80\,736$   \\
9  & Conv2DDNNLayer            & 3x3  & LReLU       & (256, 128, 27, 27)  & $93\,312$   \\
10 & DropoutLayer              &      & P=0.3       & (256, 128, 27, 27)  & $93\,312$   \\
11 & Conv2DDNNLayer            & 3x3  & LReLU       & (256, 96, 25, 25)   & $60\,000$   \\
12 & MaxPool2DDNNLayer         & 3x3  & stride 2x2  & (256, 96, 12, 12)   & $13\,824$   \\
13 & DropoutLayer              &      & P=0.4       & (256, 96, 12, 12)   & $13\,824$   \\
14 & Conv2DDNNLayer            & 3x3  & LReLU       & (256, 128, 10, 10)  & $12\,800$   \\
15 & MaxPool2DDNNLayer         & 2x2  & stride 2x2  & (256, 128, 5, 5)    & $3\,200$    \\
16 & \textbf{RotateMergeLayer} &      &             & (64, 12800)         & $12\,800$   \\
17 & DropoutLayer              &      & P=0.5       & (64, 12800)         & $12\,800$   \\
18 & DenseLayer                & 512  &             & (64, 512)           & $512$     \\
19 & FeaturePoolLayer          &      & FeaturePool & (64, 256)           & $256$     \\
20 & DropoutLayer              &      & P=0.5       & (64, 256)           & $256$     \\
21 & DenseLayer                & 512  &             & (64, 512)           & $512$     \\
22 & FeaturePoolLayer          &      & FeaturePool & (64, 256)           & $256$     \\
23 & DropoutLayer              &      & P=0.5       & (64, 256)           & $256$     \\
24 & DenseLayer                &      &             & (64, 4)             & $4$       \\ \bottomrule
\end{tabular}
\end{table}

\end{frame}



\begin{frame}\frametitle{Special layers}

\begin{columns}[t]
\begin{column}{5cm}
\centering
\par SliceRotateLayer

\vspace{0.25cm}

\begin{tikzpicture}[auto, node distance=0.1cm, >=latex]
	\node[anchor=south west,inner sep=0] (image) at (0,0) {\includegraphics[valign=c,height=1.0cm]{pics/10_left.png}};
	\begin{scope}[x={(image.south east)},y={(image.north west)}]
		\draw[orange,line width=0.4mm] (0,0.51)    rectangle (0.49,1);
		\draw[blue,line width=0.4mm]   (0.51,0)    rectangle (1,0.49);
		\draw[red,line width=0.4mm]    (0,0)       rectangle (0.49,0.49);
		\draw[green,line width=0.4mm]  (0.51,0.51) rectangle (1,1);
	\end{scope}

\end{tikzpicture}

\par $\Downarrow$

\begin{tikzpicture}
% Layers
\begin{scope}[x={(90:1cm)}, y={(0:1cm)}, z={(30:0.15cm)}]
\node[canvas is yx plane at z=0,transform shape] (imageone) at (0,0) {
	\adjincludegraphics[height=1.0cm,trim={0 {.5\width} {.5\width} 0},clip,angle=270,cfbox=blue 1pt 1pt]{pics/10_left.png}};
\node[canvas is yx plane at z=2,transform shape] at (0,0) {
	\adjincludegraphics[height=1.0cm,trim={{.5\width} {.5\width} 0 0},clip,angle=0,cfbox=green 1pt 1pt]{pics/10_left.png}};
\node[canvas is yx plane at z=4,transform shape] at (0,0) {
	\adjincludegraphics[height=1.0cm,trim={{.5\width} 0 0 {.5\width}},clip,angle=90,cfbox=orange 1pt 1pt]{pics/10_left.png}};
\node[canvas is yx plane at z=6,transform shape] at (0,0) {
	\adjincludegraphics[height=1.0cm,trim={0 0 {.5\width} {.5\width}},clip,angle=180,cfbox=red 1pt 1pt]{pics/10_left.png}};
\end{scope}
\end{tikzpicture}



\end{column}
\begin{column}{5cm}
\centering
\par RotateMergeLayer

\begin{tikzpicture}
% Layers
\begin{scope}[x={(90:1cm)}, y={(0:1cm)}, z={(30:0.15cm)}]
\node[canvas is yx plane at z=0,transform shape] (imageone) at (0,0) {
	\adjincludegraphics[height=1.0cm,trim={0 {.5\width} {.5\width} 0},clip,angle=270,cfbox=blue 1pt 1pt]{pics/10_left_cnn.png}};
\node[canvas is yx plane at z=2,transform shape] at (0,0) {
	\adjincludegraphics[height=1.0cm,trim={{.5\width} {.5\width} 0 0},clip,angle=0,cfbox=green 1pt 1pt]{pics/10_left_cnn.png}};
\node[canvas is yx plane at z=4,transform shape] at (0,0) {
	\adjincludegraphics[height=1.0cm,trim={{.5\width} 0 0 {.5\width}},clip,angle=90,cfbox=orange 1pt 1pt]{pics/10_left_cnn.png}};
\node[canvas is yx plane at z=6,transform shape] at (0,0) {
	\adjincludegraphics[height=1.0cm,trim={0 0 {.5\width} {.5\width}},clip,angle=180,cfbox=red 1pt 1pt]{pics/10_left_cnn.png}};
\end{scope}
\end{tikzpicture}

\par $\Downarrow$

%\begin{tikzpicture}[auto, node distance=0.1cm, >=latex]
\par \small Dense layer
\begin{tikzpicture}
\draw [fill=blue]   (0,0) rectangle (1,0.2);
\draw [fill=green] (1,0) rectangle (2,0.2);
\draw [fill=orange]   (2,0) rectangle (3,0.2);
\draw [fill=red] (3,0) rectangle (4,0.2);
\end{tikzpicture}
\end{column}
\end{columns}
\end{frame}




\begin{frame}\frametitle{Ordinal regression}
\begin{itemize}

\item Ordinal regression\footfullcite{ORD} is like an ordered classification.
\item Target coding: \\\vspace{0.25cm}
0 $\rightarrow$  0 0 0 0\\
2 $\rightarrow$  1 1 0 0\\
4 $\rightarrow$  1 1 1 1\\
\item Do not normalize the sigmoids in the last fully-connected layer:\\\vspace{0.5cm}
\begin{center}
\begin{Large}
$\frac{e^{-z_i}}{\sum\limits_{i=1}^Ke^{-z_i}}$ $\rightarrow$ $\frac{1}{1 + e^{-z_i}}$
\end{Large}
\end{center}
\item Use MSE loss function.
\end{itemize}
\end{frame}


\begin{frame}\frametitle{Activations}
\begin{center}
\includegraphics[valign=t,scale=0.2]{pics/relus.pdf}
\end{center}
\par We used Leaky ReLUs for convolutional layer, this activation function acts like a regularizer.
\par Used Maxout\footfullcite{MAX} activations for fully-connected layers.
\end{frame}



\begin{frame}\frametitle{Optimization I}
\begin{itemize}
\item SGD with Nesterov \footfullcite{NAG} momentum of 0.9
\item About 450000 gradient steps
\item 3 step learning rate decay: 0.02, 0.01, 0.001
\item Minibatch size of 32
\end{itemize}
\end{frame}

\begin{frame}\frametitle{Optimization II}
\begin{center}
\includegraphics[valign=t,width=\textwidth]{pics/learning_curve.png}
\end{center}
\end{frame}



\begin{frame}\frametitle{Overfitting}
\par During the whole competition we were badly overfitting.
\par Not all of the augmentation approaches worked for us, which made things worse.
\par The best thing we could come up with was an extensive dropout usage.
\end{frame}

\begin{frame}\frametitle{Decision making}
\begin{itemize}
\item Each eye was processed independently
\item A  maximum score among the eyes was assigned to both of them
\end{itemize}

\end{frame}

\begin{frame}\frametitle{Ensembling}
\par Our final submission was en ensemble of 5 neural networks.
\par The predictions we weighted according to the confusion matrix on a held-out validation set.
\par This improved the score by 0.05 points (7 \% improvement).
\end{frame}


\subsection{Software}

\begin{frame}\frametitle{}
\begin{itemize}
\item The solution was built in Python on top of Lasagne\footnote{\href{https://github.com/Lasagne/Lasagne}{https://github.com/Lasagne/Lasagne}} and Theano\footnote{\href{http://deeplearning.net/software/theano/}{http://deeplearning.net/software/theano/}}
\item Numpy, Pandas and scikit-image were used for loading and manipulating data
\item Self-written C++/OpenCV utilities for preprocessing and microaneurism detection
\item Code available at \href{https://github.com/dudevil/DRD}{https://github.com/dudevil/DRD}
\end{itemize}
\end{frame}


\subsection{Microaneurysm (MA) detection}

\begin{frame}\frametitle{Motivation I}

\par Microaneurysms are early symptoms of diabetic retinopathy
\vspace{-20pt}

\begin{table}[]
\begin{tabular}{|p{2cm}|p{8cm}|}
\hline
Disease level &  Findings observable upon dilated ophthalmoscopy \\ \hline
\footnotesize None &  No abnormalities \\ \hline
\footnotesize Mild NPDR & { Microaneurysms only} \\ \hline
\footnotesize Moderate NDPR &  More than just MA but less than severe NPDR \\ \hline
\multirow{4}{*}{\footnotesize Severe NPDR} & \multirow{4}{*}{\begin{tabular}[c]{@{}l@{}}
 $>$20 intraretinal hemorrhages in each quad \\
or  Definite venous beading in 2+ quads \\
or  Intraretinal microvascular anomalies in 1+ quad \end{tabular}} \\
 &  \\
 &  \\
 &  \\ \hline
\footnotesize PDR & \begin{tabular}[c]{@{}l@{}} Neovascularization\\  or/and Vitreous/preretinal hemorrhage\end{tabular} \\ \hline
\end{tabular}
\end{table}

\par \href{http://www.icoph.org/downloads/Diabetic-Retinopathy-Detail.pdf}{\footnotesize International Clinical Diabetic Retinopathy Disease Severity Scale, Detailed Table:  http://www.icoph.org/downloads/Diabetic-Retinopathy-Detail.pdf}

\end{frame}

\begin{frame}\frametitle{Motivation II}
\par We have problems with detection of early symptoms
\par
\begin{figure}
\begin{center}
\vspace{-10pt}
\includegraphics[width=0.4\textwidth]{pics/submission_21_inner_squares_conv5_maxout.png}
\caption{Confusion matrix on 128x128 pixels input}
\vspace{-15pt}
\end{center}
\end{figure}

\par MA have round shape with 2-5 pixels in radius on 1024x1024 image 
\par MA became invisible after downsampling to 128x128/256x256
\par $\Rightarrow$ Classes 0,1,2 almost indistinguishable due to low resolution
\par We have not enough resources\&data to learn on highres images
\par $\Rightarrow$ Let's try plain old image processing

\end{frame}

\begin{frame}\frametitle{Bag of visual words}
\par Detector: Hessian blob detector
\par Feature extractors: HOG, LBP
\par Examples of found blobs: negatives and positives
\par per-class plots of features after PCA
\par conclusion: not worked :-(
\end{frame}

\begin{frame}\frametitle{Microaneurysm candidates using the determinant of the Hessian}
\par SHORT INTRO TO HESSIAN BLOB DETECTION

\par By considering the scale-normalized determinant of the Hessian, also referred to as the 
\[ \operatorname{det} HL(x, y; t) = t^2 (L_{xx} L_{yy} - L_{xy}^2) \]
where $HL$ denotes the Hessian matrix of $L$ and then detecting scale-space maxima of this operator one obtains another straightforward differential blob detector with automatic scale selection which also responds to saddles

\[(\hat{x}, \hat{y}; \hat{t}) = \operatorname{argmaxlocal}_{(x, y; t)}(\operatorname{det} H L(x, y; t))\]
\end{frame}

\begin{frame}\frametitle{Microaneurysm candidates}
\centering
\only<1>{
\begin{tabular}{|@{}c@{}|@{}c@{}|@{}c@{}|@{}c@{}|@{}c@{}|}
\hline

Normal & Mild & Moderate & Severe & Proliferative \\

\hline
	\includegraphics[width=0.2\textwidth]{pics/classified_samples/197_left_0.jpg} &
	\includegraphics[width=0.2\textwidth]{pics/classified_samples/204_right_1.jpg} &
	\includegraphics[width=0.2\textwidth]{pics/classified_samples/82_right_2.jpg} &
	\includegraphics[width=0.2\textwidth]{pics/classified_samples/687_right_3.jpg} &
	\includegraphics[width=0.2\textwidth]{pics/classified_samples/2496_left_4.jpg} \\\noalign{\vspace{-0.15cm}}
\hline
	\includegraphics[width=0.2\textwidth]{pics/classified_samples/197_left_0_blobs.jpg} &
	\includegraphics[width=0.2\textwidth]{pics/classified_samples/204_right_1_blobs.jpg} &
	\includegraphics[width=0.2\textwidth]{pics/classified_samples/82_right_2_blobs.jpg} &
	\includegraphics[width=0.2\textwidth]{pics/classified_samples/687_right_3_blobs.jpg} &
	\includegraphics[width=0.2\textwidth]{pics/classified_samples/2496_left_4_blobs.jpg} \\

\end{tabular}
}

\only<2>{
	\includegraphics[width=0.65\textwidth]{pics/classified_samples/82_right_2.jpg}
}

\only<3>{
	\includegraphics[width=0.65\textwidth]{pics/classified_samples/82_right_2_blobs.jpg}
}
	
\end{frame}

\begin{frame}\frametitle{How blobs looks}

\begin{center}
\begin{tabular}{| b{0.15\linewidth} |@{}c@{}|@{}c@{}|@{}c@{}|@{}c@{}|@{}c@{}|}
\hline
strength & $\sigma = 1.7$ & $\sigma = 3.4$ & $\sigma =  5.1$ & $\sigma = 6.8$ & $\sigma = 8.0$ \\

\hline
$[300,450)$ & 
	\includepatches{patches_300_450_1_2_raw.pdf} & 
	\includepatches{patches_300_450_3_4_raw.pdf} & 
	\includepatches{patches_300_450_5_6_raw.pdf} & 
	\includepatches{patches_300_450_6_7_raw.pdf} & 
	\includepatches{patches_300_450_7_9_raw.pdf} \\

\hline
$[450, 600)$ & 
	\includepatches{patches_450_600_1_2_raw.pdf} & 
	\includepatches{patches_450_600_3_4_raw.pdf} & 
	\includepatches{patches_450_600_5_6_raw.pdf} & 
	\includepatches{patches_450_600_6_7_raw.pdf} & 
	\includepatches{patches_450_600_7_9_raw.pdf} \\
	
\hline
$[600, 750)$ & 
	\includepatches{patches_600_750_1_2_raw.pdf} & 
	\includepatches{patches_600_750_3_4_raw.pdf} & 
	\includepatches{patches_600_750_5_6_raw.pdf} & 
	\includepatches{patches_600_750_6_7_raw.pdf} & 
	\includepatches{patches_600_750_7_9_raw.pdf} \\

\hline
$[750, \infty)$ & 
	\includepatches{patches_750_5000_1_2_raw.pdf} & 
	\includepatches{patches_750_5000_3_4_raw.pdf} & 
	\includepatches{patches_750_5000_5_6_raw.pdf} & 
	\includepatches{patches_750_5000_6_7_raw.pdf} & 
	\\

\hline
\end{tabular}
\end{center}
\end{frame}


\begin{frame}\frametitle{How blobs looks}

\begin{center}
\begin{tabular}{| b{0.15\linewidth} |@{}c@{}|@{}c@{}|@{}c@{}|@{}c@{}|@{}c@{}|}
\hline
strength & $\sigma = 1.7$ & $\sigma = 3.4$ & $\sigma =  5.1$ & $\sigma = 6.8$ & $\sigma = 8.0$ \\

\hline
$[300,450)$ & 
	\includepatches{patches_300_450_1_2_scaled.pdf} & 
	\includepatches{patches_300_450_3_4_scaled.pdf} & 
	\includepatches{patches_300_450_5_6_scaled.pdf} & 
	\includepatches{patches_300_450_6_7_scaled.pdf} & 
	\includepatches{patches_300_450_7_9_scaled.pdf} \\

\hline
$[450, 600)$ & 
	\includepatches{patches_450_600_1_2_scaled.pdf} & 
	\includepatches{patches_450_600_3_4_scaled.pdf} & 
	\includepatches{patches_450_600_5_6_scaled.pdf} & 
	\includepatches{patches_450_600_6_7_scaled.pdf} & 
	\includepatches{patches_450_600_7_9_scaled.pdf} \\
	
\hline
$[600, 750)$ & 
	\includepatches{patches_600_750_1_2_scaled.pdf} & 
	\includepatches{patches_600_750_3_4_scaled.pdf} & 
	\includepatches{patches_600_750_5_6_scaled.pdf} & 
	\includepatches{patches_600_750_6_7_scaled.pdf} & 
	\includepatches{patches_600_750_7_9_scaled.pdf} \\

\hline
$[750, \infty)$ & 
	\includepatches{patches_750_5000_1_2_scaled.pdf} & 
	\includepatches{patches_750_5000_3_4_scaled.pdf} & 
	\includepatches{patches_750_5000_5_6_scaled.pdf} & 
	\includepatches{patches_750_5000_6_7_scaled.pdf} & 
	\\

\hline
\end{tabular}
\end{center}
\end{frame}

\begin{frame}\frametitle{BoVW preparation}
\begin{itemize}
\item Extract local descriptors from blob patch: HOG, LBP
\item K-means segmentation for quantization
\item Use histograms of visual words as feature vectors
\end{itemize}
\end{frame}

\begin{frame}\frametitle{BoVW preparation}
\par Unfortunately I got stuck on this point two weeks before challenge deadline.
\begin{center}
\begin{figure}
\includegraphics[width=0.5\textwidth]{pics/output.png}
\caption{Typical picture of BoW features  after applying PCA.}
\end{figure}
\end{center}
\end{frame}